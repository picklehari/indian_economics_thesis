\documentclass[]{article}
\usepackage{lmodern}
\usepackage{amssymb,amsmath}
\usepackage{ifxetex,ifluatex}
\usepackage{fixltx2e} % provides \textsubscript
\ifnum 0\ifxetex 1\fi\ifluatex 1\fi=0 % if pdftex
  \usepackage[T1]{fontenc}
  \usepackage[utf8]{inputenc}
\else % if luatex or xelatex
  \ifxetex
    \usepackage{mathspec}
  \else
    \usepackage{fontspec}
  \fi
  \defaultfontfeatures{Ligatures=TeX,Scale=MatchLowercase}
\fi
% use upquote if available, for straight quotes in verbatim environments
\IfFileExists{upquote.sty}{\usepackage{upquote}}{}
% use microtype if available
\IfFileExists{microtype.sty}{%
\usepackage{microtype}
\UseMicrotypeSet[protrusion]{basicmath} % disable protrusion for tt fonts
}{}
\usepackage[margin=1in]{geometry}
\usepackage{hyperref}
\hypersetup{unicode=true,
            pdftitle={Economic Growth post Independence},
            pdfauthor={Hariraj K},
            pdfborder={0 0 0},
            breaklinks=true}
\urlstyle{same}  % don't use monospace font for urls
\usepackage{graphicx,grffile}
\makeatletter
\def\maxwidth{\ifdim\Gin@nat@width>\linewidth\linewidth\else\Gin@nat@width\fi}
\def\maxheight{\ifdim\Gin@nat@height>\textheight\textheight\else\Gin@nat@height\fi}
\makeatother
% Scale images if necessary, so that they will not overflow the page
% margins by default, and it is still possible to overwrite the defaults
% using explicit options in \includegraphics[width, height, ...]{}
\setkeys{Gin}{width=\maxwidth,height=\maxheight,keepaspectratio}
\IfFileExists{parskip.sty}{%
\usepackage{parskip}
}{% else
\setlength{\parindent}{0pt}
\setlength{\parskip}{6pt plus 2pt minus 1pt}
}
\setlength{\emergencystretch}{3em}  % prevent overfull lines
\providecommand{\tightlist}{%
  \setlength{\itemsep}{0pt}\setlength{\parskip}{0pt}}
\setcounter{secnumdepth}{0}
% Redefines (sub)paragraphs to behave more like sections
\ifx\paragraph\undefined\else
\let\oldparagraph\paragraph
\renewcommand{\paragraph}[1]{\oldparagraph{#1}\mbox{}}
\fi
\ifx\subparagraph\undefined\else
\let\oldsubparagraph\subparagraph
\renewcommand{\subparagraph}[1]{\oldsubparagraph{#1}\mbox{}}
\fi

%%% Use protect on footnotes to avoid problems with footnotes in titles
\let\rmarkdownfootnote\footnote%
\def\footnote{\protect\rmarkdownfootnote}

%%% Change title format to be more compact
\usepackage{titling}

% Create subtitle command for use in maketitle
\newcommand{\subtitle}[1]{
  \posttitle{
    \begin{center}\large#1\end{center}
    }
}

\setlength{\droptitle}{-2em}
  \title{Economic Growth post Independence}
  \pretitle{\vspace{\droptitle}\centering\huge}
  \posttitle{\par}
  \author{Hariraj K}
  \preauthor{\centering\large\emph}
  \postauthor{\par}
  \predate{\centering\large\emph}
  \postdate{\par}
  \date{March 6, 2018}


\begin{document}
\maketitle

\subsection{India: A case study}\label{india-a-case-study}

The mid 20th century saw formation of new democratic nations
worldwide.It has also seen countries become world leaders and rapid
development like never before, even though most nations saw
industrialization as the ideal way to economic development and
prosperity.

\subsection{India GDP over the years}\label{india-gdp-over-the-years}

\begin{verbatim}
##       Year GDP in Crores
## 1  1951-52        286147
## 2  1952-53        294267
## 3  1953-54        312177
## 4  1954-55        325431
## 5  1955-56        333766
## 6  1956-57        352766
## 7  1957-58        348500
## 8  1958-59        374948
## 9  1959-60        383153
## 10 1960-61        410279
## 11 1961-62        423011
## 12 1962-63        431960
## 13 1963-64        453829
## 14 1964-65        488247
## 15 1965-66        470402
## 16 1966-67        475190
## 17 1967-68        513860
## 18 1968-69        527270
## 19 1969-70        561630
## 20 1970-71        589787
## 21 1971-72        595741
## 22 1972-73        593843
## 23 1973-74        620872
## 24 1974-75        628079
## 25 1975-76        684634
## 26 1976-77        693191
## 27 1977-78        744972
## 28 1978-79        785965
## 29 1979-80        745083
## 30 1980-81        798506
## 31 1981-82        843426
## 32 1982-83        868092
## 33 1983-84        936270
## 34 1984-85        973357
## 35 1985-86       1013866
## 36 1986-87       1057612
## 37 1987-88       1094993
## 38 1988-89       1206243
## 39 1989-90       1280228
## 40 1990-91       1347889
## 41 1991-92       1367171
## 42 1992-93       1440504
## 43 1993-94       1522344
## 44 1994-95       1619694
## 45 1995-96       1737741
## 46 1996-97       1876319
## 47 1997-98       1957032
## 48 1998-99       2087828
## 49 1999-2K       2246276
## 50 2000-01       2342774
## 51 2001-02       2472052
## 52 2002-03       2570690
## 53 2003-04       2777813
## 54 2004-05       2971464
## 55 2005-06       3253073
## 56 2006-07       3564364
## 57 2007-08       3896636
## 58 2008-09       4158676
## 59 2009-10       4516071
## 60 2010-11       4937006
## 61 2011-12       5243582
\end{verbatim}

\subsection{Indian GDP share}\label{indian-gdp-share}

Pre-Independent India remained an agriculture based nation.However,
there has been a rather steep decline in the agriculture based labour
force. Census data indicated that the percentage of farmers went down
from 40\% of the total population in 2001 to 33\% in 2011. However, the
IT boom in India has resulted in significant increase in the Service
sector. Interestingly enough, the growth of the Service sector remains
near linear with negative slope with respect to the increase in the
Agriculture Sector.

\includegraphics{Paper_files/figure-latex/plot_a-1.pdf}
\includegraphics{Paper_files/figure-latex/plot_b-1.pdf}

\subsection{Sector wise growth of Indian
GDP}\label{sector-wise-growth-of-indian-gdp}

India being a nation with diversity in lifestyle(s), it becomes evident
that the GDP growth over the years are erratic, and the result of
several factors.

\includegraphics{Paper_files/figure-latex/plot_c-1.pdf} \#\# India and
the World Though India remains to be one of the wealthiest countries in
the world, its large population leads to major income disparity. The
nation remained a closed economy for most it's time in the past. Yet,
the liberalization of the economy in 1990's had a huge impact on the
Indian economy. The economy of the nation had, ever since then, a growth
rate more significant than that of the total world GDP.
\includegraphics{Paper_files/figure-latex/plot_d-1.pdf} \#\# Conclusion
Although the Indian approach towards finance managment tells a success
story, other factors such as high population and high income
inequalities remain a barrier India has to surpass to become a world
superpower. Plotting real per capita GDP we infer that India still has a
long way to go to achieve economic prosperty.The plot below shows the
Real GDP of countries.
\includegraphics{Paper_files/figure-latex/plot_e-1.pdf} \#\#\# Sources -
\href{data.gov.in}{Govt. of India open data} -
\href{ourworldindata.org}{Our World in data}


\end{document}
